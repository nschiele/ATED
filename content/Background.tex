\section{Background}
\label{sec:background}

\begin{figure*}
    \includegraphics[width=\linewidth]{img/TargetAT.png}
    \caption{An attack tree adapted from Naik \etal~\cite{naikEvaluationPotentialAttack2022} that is used in the study described in Section~\ref{sec:methodology}. }
    \label{fig:tartgetAT}
\end{figure*}

We define attack trees to be a rooted acyclic structure with the following recursive definition adapted from Gadyatskaya~\etal~\cite{gadyatskayaRefinementAwareGenerationAttack2017}.

\begin{definition} \label{def:attack-tree} An attack tree with $n$ nodes is defined as $T = t\Delta(t_0,...,t_i)$. Where $t = b|t\Delta(t_0,...,t_i)$

    % We define the $i\text{th}$ node according to left-right post-order number to be the following, $T[i] = b\Delta(T[j],...,T[k])$. 

    We give $b$ to be some action within the attack scenario, which is the node label of $t$. For clarity, we additionally refer to this label as $t.\text{label}$. We give $\Delta$ to be the refinement $\Delta = \AND|\OR|\SAND$. For clarity and consistency, we refer to the refinement of a given node to be $t.\Delta$. Following the refinement, we have a list of nodes which are the children of $t$, defined as $t_0,...,t_i$ which is the left to right order of the children (regardless of if order is significant). We refer to the list of children as $t.\text{children}$. This list of nodes can be empty, in the case of leaf nodes.
\end{definition}
