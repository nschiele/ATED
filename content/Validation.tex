\section{Validation}
\label{sec:validation}

In order to compare any approach to comparing attack trees, we must first establish a means of validating the results. We propose two methods of validating the distance measures:

\textbf{Theoretical Validation}. This validation method involves defining a series of basic transformation examples (BTEs) which represent all of the most fundamental ways two attack trees could differ from each other. These BTEs can be adapted to only express the transformations that we would expect to see, or that we explicitly would desire to compare two distance measures along. For example, if we only wish to evaluate distance measures according to their ability to compare leaf nodes, we could define a set of BTEs which only show changes in leaf nodes. If we want a more comprehensive comparison, we can define a complete set of BTEs which show all possible basic transformations, which is what we do in Section~\ref{sec:methodology}. We argue that if any transformation can be expressed as a series of BTEs, then we can validate the distance measure by comparing the distance measure's output to the expected distance of each BTE.

\textbf{Experimental Validation}. We offer an experimental design which can be used to validate a distance measure on ``real-world'' data. This requires a dataset which is two fold. First, we require a base set of trees which should all be identical. This can be created by asking participants to create the same tree. Any variation between the trees would likewise be representative of ``real-world'' label variation, and not due to difference in data. In this way, we can evaluate how the distance measures handle similar but not identical data. Second, we require the same set of trees to be altered along similar circumstances. This results in a second set of trees such that the first set is a complete subset of the second set. This allows us to evaluate the behavior of the distance measures when comparing a tree to a known subset of that tree. This behavior should be predictable, and we can access distance measures if they display behavior we expect. We describe the details of our experiment in Section~\ref{sec:methodology}.