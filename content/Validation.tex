\section{Validation}
\label{sec:validation}

In order to evaluate any approach for comparing attack trees, we must first establish a means of validating the results and assessing the quality of this approach. We propose two methods of validating the distance measures:

\textbf{Theoretical validation}. This validation method involves defining a series of basic transformation examples (BTEs) that represent the most fundamental ways in which two attack trees could differ from each other. These BTEs can be adapted to only express the transformations that we would expect to see, or that we explicitly would desire to compare two distance measures along. For example, if we only wish to evaluate distance measures according to their ability to compare leaf nodes (as the traditional attack tree semantics expect~\cite{mauwFoundationsAttackTrees2006}), we could define a set of BTEs which only shows changes in leaf nodes. For a more comprehensive comparison, we can define a larger set of BTEs to show all possible basic transformations. This is what we do in Section~\ref{sec:methodology}. We argue that if any transformation can be expressed as a series of BTEs, then we can validate the distance measure by comparing the distance measure's output to the expected distance of each BTE.

\textbf{Experimental validation}. We offer an experimental design that can be used to validate a distance measure on ``real-world'', human-produced data. This requires a dataset that contains two sets of attack tree models. First, we require a base set of trees which should all be identical. This can be created by asking participants to create an attack tree describing the same precisely defined scneario. Any variation between the trees would likewise be representative of ``real-world'' label variation, and not due to differences in attack steps identified by distinct experts. In this way, we can evaluate how the distance measures handle similar but not fully identical data. 

Second, we require the base set of trees to be slightly expanded by the authors. This results in a second set of trees that are expected to be slightly different, but not very diverse. At the same time, the base set contains attack trees that are subtrees of this second set. This allows us to evaluate the behavior of the distance measures when comparing a tree to a known subset of that tree, and to compare a set of attack trees that are slightly different from each other. This behavior should be predictable, and we can access distance measures if they display the behavior we expect. We describe the details of our study design and the data collection process in Section~\ref{sec:methodology}.