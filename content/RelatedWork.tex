



\section{Related Work}
\label{sec:related-work}

Distance between data structures is not a new concept. Many works have explored the idea of ``distance'' between strings. In string edit distance, where the difference in strings is given my a min-cost path taken by either adding a character, removing a character or replacing a character. By fining the minimum cost needed to transform one string into another, a ``distance'' value can be given. As the higher cost a transformation, the further apart two strings must be \NS{cite all the string edit distance papers}.

Tree edit distance is seen as an extension of the string edit distance problem. Tai first tackled the problem, suggesting an edit distance metric for two directed acyclic graphs (DAG)s \cite{tai_tree--tree_1979}. Zhang and Shasha have written the seminal work on tree edit distance~\cite{zhang_simple_1989}. In their work, they describe a simple algorithm for calculating the distance between two trees. This algorithm is based on the idea of a \textit{forest} distance, which is the distance between two forests. A forest is possible disjoint a collection of trees, though a forest can consist of a single tree. The distance between two forests is the minimum cost of transforming one forest into another. This is calculated by finding the minimum cost of transforming each tree in the first forest into each tree in the second forest. This is done recursively until the minimum cost of transforming each node in the first tree into each node in the second tree is found. The minimum cost of transforming the first tree into the second tree is then given as the optimal \emph{tree edit distance} between two trees. This edit distance is given as both a value, the cost of the sequence of edits, as well as the sequence of edits itself. This sequence of edits has 3 possible operations.

Most of the research and development on tree edit distance focuses calculation optimization. As shown by Zhang~\etal, the tree edit distance problem for unordered trees is an \textit{NP}-Complete problem~\cite{zhang_editing_1992}. As such, the development of novel optimal calculation strategies is necessary to enable comparison of larger tree structures. Yoshino \etal\ developed a dynamic programming A$^*$ algorithm for computing unordered tree edit distance (UTED), which offer significant performance gains over exhaustive search~\cite{yoshino_dynamic_2013}. Their methodology uses an A$^*$ algorithm to construct a search tree of mappings. The distance is then calculated from these mappings. Unfortunately, these mapping rely on an absolute equivalence between nodes to establish a mapping, which is not practical or particularly useful for the labels of attack trees, which as such cannot be used for attack tree distance.
\NS{However, the instutition to optimize finding mappings and then find the distance based on the mappings is a useful methodology that we can apply to our work.}
Other work based on the A$^*$ algorithm methodology, such as the optimizations offered by Paaßen~\cite{paasen_-algorithm_2021} for computing UTED will have the same unsuitability.

McVicar~\etal\ have developed a method of calculating UTED in SuMoTED by focusing on allowing nodes to move up or down a tree and organizing edits around a consensus tree~\cite{mcvicar_sumoted_2016}. This methodology allows them to achieved polynomial time calculation of UTED. However, application of this methodology to attack trees presents difficult. Namely, the same issue as the A$^*$ algorithm, the requirement of exact equivalence between node labels.

% McVicar~\etal\ have developed a method of calculating UTED in SuMoTED by focusing on allowing nodes to move up or down a tree and organizing edits around a consensus tree~\cite{mcvicar_sumoted_2016}. This methodology allows them to achieved polynomial time calculation of UTED. However, application of this methodology to attack trees presents difficult. Namely, the same disallowance of ordered nodes. Additionally, by allowing movement of nodes below leaf nodes would functionally introduce a refinement to the attack tree, which would require a new mechanism to accomodate this potential action. In effect, unlike SuMoTED, movement of nodes up a tree would not be equivalent to movement of nodes down a tree, and this equivalence is assumed in SuMoTED.

Pawlick and Augusten proposed RTED, a more optimal method of calculating ordered TED (OTED)~\cite{pawlik_rted_2011}. Their methodology is faster than Zhang and Shasha, but specifically for larger trees (500+ nodes). Attack trees by contrast cannot grow this large, and previous work has found that roughly 50 nodes would be an exceptionally large for attack trees \NS{I found this in acceptability 1}. As such, RTED, while an important optimization and contribution, does not offer a significant advantage for attack trees over Zhang and Shasha.

Tree edit distance, like string edit distance, has a wide array of applications. Just as string edit distance has been used to compare sequences of DNA \NS{cite},






% Much of the work in this field has built upon work by Kuo-Chung Tai who suggested that the distance between trees is similar to several previous works comparing the differences in strings~\cite{tai_tree--tree_nodate}.


% In the Zhang and Shasha algorithm the three possibilities for an edit operation between individual nodes are, (i) a node must be added, (ii) a node must be removed, or (iii) a node must be replaced~\cite{zhang_simple_1989}. Each of these operations has a cost

% Zhang and Shasha proposed a commonly cited simple algorithm for calculating tree edit distance~\cite{zhang_simple_1989}. We use this algorithm in this paper as it is a common strategy for implementing and testing extensions to tree edit distance. As such, optimizations based on the Zhang and Shasha algorithm can be applied to our methodology as we show in section \NS{When I write that section, I'll reference it here}.