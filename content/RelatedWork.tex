\section{Background}

Distance between data structures is not a new concept. Many works have explored the idea of ``distance'' between strings. In string edit distance, where the difference in strings is given my a min-cost path taken by either adding a character, removing a character or replacing a character. By fining the minimum cost needed to transform one string into another, a ``distance'' value can be given. As the higher cost a transformation, the further apart two strings must be \NS{cite all the string edit distance papers}.

Much of the work in this field has built upon work by Kuo-Chung Tai who suggested that the distance between trees is similar to several previous works comparing the differences in strings~\cite{tai_tree--tree_nodate}.

\section{Related Work}

Tree Edit Distance is not a new computational challenge. However, most of the development on tree edit distance focuses calculation optimization. As shown by Zhang~\etal, the tree edit distance problem for unordered trees is an \textit{NP}-Complete problem~\cite{zhang_editing_1992}. As such, the develop of novel optimal calculation strategies is necessary to enable comparison of larger tree structures. Zhang and Shasha proposed a commonly cited simple algorithm for calculating tree edit distance~\cite{zhang_simple_1989}. We use this algorithm in this paper as it is a common strategy for implementing and testing extensions to tree edit distance. As such, optimizations based on the Zhang and Shasha algorithm can be applied to our methodology as we show in section \NS{When I write that section, I'll reference it here}.

Tree edit distance, like string edit distance, has a wide array of applications. Just as string edit distance has been used to compare sequences of DNA \NS{cite},
