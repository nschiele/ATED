



% \section{Proofs}
% \label{appendix:proofs}
% \subsection{Proof of Lemma~\ref{lem:gamma-delta}}
% \label{appendix:lem:gamma-delta}
% \begin{proof}


%     \begin{enumerate}
%         \item In the case of node removal, there can be no additional cost for changing a refinement. That is, if we remove a node $\ATnode{d}{i}$ from $T$, then we have:

%               $$\gamma(\ATnode{d}{i} \rightarrow {\Lambda})$$

%               $\Lambda$ is an empty tree, and by definition does not contain any refinements. Therefore, the cost of changing the refinement is zero.

%         \item In the case of adding a node, the cost of adding a refinement would be included in the cost of adding the node. That is, if we add a node $\ATnode{d}{i}$ to $T$, then we have:

%               $$\gamma(\Lambda \rightarrow {\ATnode{d}{i}})$$

%               It is not possible to for a node in an attack tree to not have a refinement. If we separate the cost of adding a node and the cost of adding a refinement, we have one of the two following cases:
%               \begin{enumerate}
%                   \item A node is added without a refinement, which gives a refinement addition cost of 0, but results in an attack tree which is not valid given our attack tree definition.
%                   \item The cost of adding a refinement is \textbf{always} added to the cost of adding a node, which results in the new cost of adding a node to always include $\gamma(\Delta)$.
%               \end{enumerate}

%               Given one of these cases results in an invalid tree, the other case must always apply. Therefore, by convention, we do not separate the cost of adding a node and the cost of adding a refinement, these are one and the same.

%         \item In the case of replacing a node, the cost of replacing a refinement would be:

%               $$\gamma({\ATnode{d}{i}} \rightarrow {\ATnode{e}{j}})$$

%               Which we declare to consist of the sum following two costs:

%               $$\gamma({\ATlabel{d}{i}} \rightarrow {\ATlabel{e}{j}})$$

%               Which is the cost of changing one label to another. This is the original cost of replacing a node according to Zhang-Shasha. We also have:

%               $$\gamma(\Delta)$$

%               Which as previously stated is the cost of changing a refinement.



%     \end{enumerate}

% \end{proof}

% \subsection{Proof of Lemma~\ref{lem:gamma-delta-2}}
% \label{appendix:lem:gamma-delta-2}

% \begin{proof}
%     Let $T$ be an attack tree.

%     Assume that $\gamma(\Delta) > \gamma(\ATnode{e}{j} \rightarrow {\Lambda}) + \gamma(\Lambda \rightarrow {\ATnode{e}{j}})$.

%     Let $S$ be the optimal sequence of edit operations according to the Zhang-Shasha algorithm. That is, $\gamma(S)$ is minimal for all possible edit sequences for $\delta(T_1, T_2)$ Let some operation $s \in S$ be an operation to replace some node, $\ATnode{d}{i}$, with another, $\ATnode{e}{j}$.

%     Thus, $\gamma(s) = \gamma({\ATlabel{d}{i}} \rightarrow {\ATlabel{e}{j}}) + \gamma(\Delta)$

%     We have two cases:

%     \begin{enumerate}
%         \item $\ATnode{d}{i}.\Delta = \ATnode{e}{j}.\Delta$

%               In this case, both $\ATnode{d}{i}$ and $\ATnode{e}{j}$ have the same refinement. Thus, $\gamma(\Delta) = 0$. Therefore, $\gamma(s) = \gamma({\ATlabel{d}{i}} \rightarrow {\ATlabel{e}{j}})$.

%         \item $\ATnode{d}{i}.\Delta \ne \ATnode{e}{j}.\Delta$

%               In this case, both $\ATnode{d}{i}$ and $\ATnode{e}{j}$ have different refinements. Thus, $\gamma(\Delta) > 0$. Therefore, $\gamma(s) = \gamma({\ATlabel{d}{i}} \rightarrow {\ATlabel{e}{j}}) + \gamma(\Delta)$.

%               However, we have assumed that $\gamma(\Delta) > \gamma(\ATnode{e}{j} \rightarrow {\Lambda}) + \gamma(\Lambda \rightarrow {\ATnode{e}{j}})$. Therefore, $\gamma(s) = \gamma({\ATlabel{d}{i}} \rightarrow {\ATlabel{e}{j}}) + \gamma(\Delta) > \gamma(\ATnode{e}{j} \rightarrow {\Lambda}) + \gamma(\Lambda \rightarrow {\ATnode{e}{j}})$, as by convention $\gamma$ cannot result in a negative value.

%               As such, we can replace $s$ with the sequence of operations $s_1$ and $s_2$, where $s_1$ is the operation to remove $\ATnode{d}{i}$ and $s_2$ is the operation to add $\ATnode{e}{j}$. Thus, $\gamma(s_1) = \gamma(\ATnode{e}{j} \rightarrow {\Lambda})$ and $\gamma(s_2) = \gamma(\Lambda \rightarrow {\ATnode{e}{j}})$. Therefore, $\gamma(s_1) + \gamma(s_2) < \gamma(s)$.

%               This results in a contradiction, as $S$ is the optimal sequence of edit operations according to the Zhang-Shasha algorithm, it must not be possible to replace any $s \in S$ with an operation, or sequence of operations, with lower cost.
%     \end{enumerate}

%     Therefore, if $\gamma(\Delta) > \gamma(\ATnode{e}{j} \rightarrow {\Lambda}) + \gamma(\Lambda \rightarrow {\ATnode{e}{j}})$, then $\gamma(\Delta)$ either must be 0 for a change node edit operation to be included in the optimal sequence of edit operations (case 1), or the optimal sequence of edit operations must always result in node removal then replacement (case 2). In both cases, $\gamma(\Delta)$ is not used.

%     Therefore, in order to include the cost of changing refinements in the cost of replacing a node, it must be the case that $\gamma(\Delta) \le \gamma(\ATnode{e}{j} \rightarrow {\Lambda}) + \gamma(\Lambda \rightarrow {\ATnode{e}{j}})$.


% \end{proof}














\subsection{Label Distance Algorithm}
\label{appendix:alg:label-distance}
\begin{algorithm}[H]
    \caption{An algorithm to calculate the label distance between two attack trees.}
    \label{alg:label-distance}
    \begin{algorithmic}
        \State Two attack trees $T_1$ and $T_2$ according to Definition~\ref{def:attack-tree} with $a$ and $b$ total nodes respectively
        \State $M$ is the set of mappings between nodes in $T_1$ and $T_2$
        \State $A$ is the list of node labels in $T_1$
        \State $B$ is the list of node labels in $T_2$
        \State $d$ is the distance between attack trees
        \State $M \gets \emptyset$
        \State Let $L$ be the $a \times b$ matrix of semantic similarity values between labels in $A$ and $B$
        \While{$L$ is not empty}
        \State Find the maximum value, $\delta$, in $L$ at index $i, j$
        \State Remove row $i$ and column $j$ from $L$
        \If{$\delta > \epsilon$}
        \State Add $(A[i], B[j], \delta)$ to $M$
        \Else
        \State Add $(A[i], \Lambda, 0)$ to $M$
        \State Add $(\Lambda, B[j], 0)$ to $M$
        \State $d = d + 1$
        \EndIf
        \State Remove $A[i]$ from $A$
        \State Remove $B[j]$ from $B$
        \EndWhile
        \For{each $a \in A$}
        \State Add $(a, \Lambda, 1)$ to $M$
        \State $d = d + 1$
        \EndFor
        \For{each $b \in B$}
        \State Add $(\Lambda, b, 1)$ to $M$
        \State $d = d + 1$
        \EndFor
        \State \Return $d$, $M$
    \end{algorithmic}
\end{algorithm}

\subsection{Radical Distance Algorithm}
\label{appendix:alg:radical-distance}
\begin{algorithm}[H]
    \caption{An algorithm to compute radical distance}
    \label{alg:recursive-radical}
    \begin{algorithmic}
        \State Two attack trees $T_1$ and $T_2$ according to Definition~\ref{def:attack-tree} with $a$ and $b$ total nodes respectively
\State $D_1$, $D_2$ $\gets$ the radical dictionary according to the decomposition in \cite{schiele2021novel} for $T_1$ and $T_2$, respectively
        \State $M$ $\gets$ the mapping between $D_1$ and $D_2$ indexed by radical root nodes according to semantic similarity (from semantic label distance)
        \State $d \gets 0$
        \For{$m \in M$, where $m = (\ATnode{d}{i}, \ATnode{e}{j})$ and $\ATnode{d}{i}, \ATnode{e}{j}$ are indices for $D_1$, and $D_2$, respectively}
        \If {$\delta(\ATnode{d}{i}, \ATnode{e}{j}) < \epsilon$}
        \State $d \gets d + 1$
        \EndIf
        \If {$Delta(\ATnode{d}{i}) \ne Delta(\ATnode{e}{j})$ and $\ATnode{d}{i}, \ATnode{e}{j} \ne \Lambda$}
        \State $d \gets d + 0.5$
        \EndIf
        \State $M_c \gets$ the semantic mappings (from semantic label distance) between child$(\ATnode{d}{i})$ and child$(\ATnode{e}{j})$
        \For{$c \in M_c$ where  $c = (\ATnode{d+1}{p}, \ATnode{e+1}{q})$}
        \If{$\ATnode{d+1}{p} \not\in D_1$ and $\ATnode{e+1}{p} \not\in D_2$
            \If $\delta(\ATnode{d+1}{p}, \ATnode{e+1}{q}) < \epsilon$}
        \State $d \gets d + 1$
        \EndIf
        \EndIf
        \EndFor
        \EndFor
        \State \Return $d$
    \end{algorithmic}
\end{algorithm}
















% \section{Operations table for counterexamples}
\begin{table*}
    \resizebox{\textwidth}{!}{
        \begin{tabular}{lcccccccccccccccc}
            \toprule
            \multirow{2}{*}{\shortstack[l]{Basic Transformation                                                                                                  \\Example (BTE)}}      & \multicolumn{4}{|c|}{Label Distance} & \multicolumn{4}{|c|}{Tree Edit Distance} & \multicolumn{4}{|c|}{Radical Distance} & \multicolumn{4}{|c|}{Multiset Distance}                                                                                                 \\
                                 & Remove & Add & Change & Match & Remove & Add & Change & Match & Remove & Add & Change & Match & Remove & Add & Change & Match \\
            \midrule
            Order Reversed       & 0      & 0   & 0      & 7     & 0      & 0   & 6      & 1     & 0      & 0   & 0      & 7     & 0      & 0   & 0      & 4     \\
            Refinement Switch    & 0      & 0   & 0      & 7     & 0      & 0   & 2      & 5     & 0      & 0   & 2      & 5     & 1      & 1   & 1      & 2     \\
            Extra Intermediate   & 1      & 0   & 0      & 7     & 1      & 0   & 0      & 7     & 1      & 0   & 0      & 7     & 0      & 0   & 0      & 4     \\
            Missing Intermediate & 0      & 1   & 0      & 6     & 0      & 1   & 0      & 6     & 1      & 3   & 0      & 4     & 0      & 0   & 0      & 4     \\
            Extra Leaf           & 1      & 0   & 0      & 7     & 1      & 0   & 0      & 7     & 1      & 0   & 0      & 7     & 1      & 0   & 0      & 4     \\
            Missing Leaf         & 0      & 1   & 0      & 6     & 0      & 1   & 0      & 6     & 0      & 1   & 0      & 6     & 0      & 1   & 0      & 3     \\
            Changed Root         & 0      & 0   & 1      & 6     & 0      & 0   & 1      & 6     & 0      & 0   & 1      & 6     & 0      & 0   & 0      & 4     \\
            Changed Intermediate & 0      & 0   & 1      & 6     & 0      & 0   & 1      & 6     & 0      & 0   & 1      & 6     & 0      & 0   & 0      & 4     \\
            Changed Leaf         & 0      & 0   & 1      & 6     & 0      & 0   & 1      & 6     & 0      & 0   & 1      & 6     & 0      & 0   & 1      & 3     \\
            Move Adjacent        & 0      & 0   & 0      & 7     & 1      & 1   & 0      & 6     & 1      & 1   & 0      & 6     & 1      & 2   & 0      & 2     \\
            Move Up              & 0      & 0   & 0      & 7     & 1      & 1   & 0      & 6     & 1      & 1   & 0      & 6     & 0      & 0   & 0      & 4     \\
            Move Down            & 0      & 0   & 0      & 7     & 1      & 1   & 0      & 6     & 2      & 1   & 0      & 5     & 0      & 1   & 0      & 3     \\
            \bottomrule
        \end{tabular}

    }
    \caption{Table showing the operations per BTE and distance measure}
\end{table*}






\subsection{Experiment Questions}
\label{app:exp-questions}


% \subsection*{ADT 1: Assembling ADTs}

% The following attack \textbf{leaf} nodes are provided. The overall goal of this scenario (and thus the root node of the tree) is \textbf{Rob bank}. Assemble an attack-defense tree using these leaf nodes. Do not add any additional leaf nodes. You may add any intermediary nodes you wish.

% \textbf{Attack leaf nodes:}
% % Why the fuck is there so much space here?
% % \vspace{-10cm}
% % \begin{itemize}
% %   \item Hire Outright
% %   \item Promise part of the stolen money
% %   \item Threaten insiders

% %   \item Buy tools
% %   \item Steal tools
% %   \item Gain Access
% %   \item Walk through front door
% %   \item Locate start of tunnel
% %   \item Find direction to tunnel
% % \end{itemize}

% Hire Outright, Promise part of the stolen money, Threaten insiders, Buy tools, Steal tools, Gain Access, Walk through front door, Locate start of tunnel, Find direction to tunnel

% \textbf{Defense leaf nodes:}

% Personnel Risk Management, Check employee financial situation


% \subsection*{Perception Questions}


% \subsubsection*{Likert Questions}
% \begin{itemize}
%   \setlength{\itemindent}{\qIndent}
%   \item[\surveyq{LS-ADT1-L1}] I find the structure of attack tree easy to understand
%   \item[\surveyq{LS-ADT1-L2}] Given all the nodes of an attack tree, it is easy for me to assemble the tree
%   \item[\surveyq{LS-ADT1-L3}] Given only the leaf nodes of an attack tree, it is easy for me to assemble the tree.
%   \item[\surveyq{LS-ADT1-L4}] I would rather define my own intermediary nodes
%   \item[\surveyq{LS-ADT1-L5}] The process of assembling the attack tree helped me better understand the attack scenario.
% \end{itemize}

% \subsubsection*{Short Response Questions}
% \begin{itemize}
%   \setlength{\itemindent}{\qIndent}
%   \item[\surveyq{LS-ADT1-W1}] What did you find most difficult about this task? Why?
%   \item[\surveyq{LS-ADT1-W2}] How did you go about solving this task? What was your methodology?
% \end{itemize}



% \subsection*{ADT 2: Building ADTs}

% The following text scenario is provided for you. Please create a complete attack defense tree \textbf{of this scenario}. \textbf{Do not add extra information that is not in the scenario}. Try to encapsulate the entire scenario with an attack-defense tree (don't leave any aspect of the attack scenario out).

% \emph{Scenario:} 
% The goal is to open a safe. To open the safe, an attacker can pick the lock,
% learn the combination, cut open the safe, or install the safe improperly so
% that he can easily open it later. Some models of safes are such that they cannot be picked, so if this model is used, then an attacker is unable to pick the lock. There are also auditing services to check if safes and other security technology is installed correctly. To learn the combination, the attacker
% either has to find the combination written down or get the combination
% from the safe owner. If the password is such that the safe owner can remember it, then the safe owner would not need to write it down.



% \subsection*{Perception Questions}

% \subsubsection*{Likert Questions}
% \begin{itemize}
%   \setlength{\itemindent}{\qIndent}
%   \item[\surveyq{LS-ADT2-L1}] I prefer reading attack trees to text descriptions of attacks.
%   \item[\surveyq{LS-ADT2-L2}] The process of building the attack tree helped me better understand the attack scenario.
% \end{itemize}

% \subsubsection*{Short Response Questions}
% \begin{itemize}
%   \setlength{\itemindent}{\qIndent}
%   \item[\surveyq{LS-ADT2-W1}] What did you find most difficult about this task? Why?
%   \item[\surveyq{LS-ADT2-W2}] How did you go about building the ADT?\@ What was your methodology?
%   \item[\surveyq{LS-ADT2-W3}] What was the first node you added to your tree?
% \end{itemize}


% \subsection*{ADT 3: Using Attack Trees}
This question is slightly different than the other questions. You need to create attack trees only. \textbf{This means you should NOT use defense nodes at all for this question}. In Part I, raw an attack tree from the provided scenario; \textbf{include all information from the scenario, do not include information that is not in the scenario}. In Part II, add onto the provided attack tree with new nodes and refinements that you find through your own research.

\subsection*{AT1: Attack tree from scenario}

\emph{Scenario:}  Many attackers aim to obtain personal data. Gathering personal data can be completed through unauthorized access to profile, credential creep, or a background data attack. Unauthorized access to profile requires gaining user credentials and accessing the profile. The credentials can be gained through a malware attack or a social engineering attack, and the profile can be accessed by stealing a phone or by remote access. Credential creep can be completed by submitting a request for additional data other than what is needed for verification or by user profiling. Finally, a background data attack requires both obtaining a sensitive dataset and linking the dataset via a request for verification. 


\subsection*{AT2: Finding new attack components}

Doing your own research, add at least 5 new nodes and 2 new refinements to the attack tree you created in the previous section.





\subsection*{Perception Questions}

\subsubsection*{Likert Questions}
\begin{enumerate}
    \setlength{\itemindent}{\qIndent}
  \item[\surveyq{LS-ADT3-L1}] I prefer reading attack trees to text descriptions of attacks.
  \item[\surveyq{LS-ADT3-L2}] The process of building the attack tree helped me better understand the attack scenario.
  \item[\surveyq{LS-ADT3-L3}]  The ADT communicates the attack scenario better than the written scenario.
  \item[\surveyq{LS-ADT3-L4}] Using the ADT Web App made this task easier than if I had done it by hand.
\end{enumerate}

\subsubsection*{Short Response Questions}
\begin{enumerate}
    \setlength{\itemindent}{\qIndent}
  \item[\surveyq{LS-ADT3-W1}] What did you find most difficult about this task? Why?
  \item[\surveyq{LS-ADT3-W2}] How did you go about building the ADT? What was your methodology?
  \item[\surveyq{LS-ADT3-W3}] What was the first node you added to your tree?
  \item[\surveyq{LS-ADT3-W4}]How would you describe using the ADT Web App? What aspects of the app made this task easier? What aspects made this task harder?
\end{enumerate}

% \subsection*{ADT 4: Creating ADTs}

% Construct an attack defense tree of a scenario of your choice. Your tree should be complete (covers all reasonable attack scenarios) and reasonably large.


% \subsection*{Perception Questions}

% \subsubsection*{Likert Questions}
% \begin{itemize}
%   \setlength{\itemindent}{\qIndent}
%   \item[\surveyq{LS-ADT4-L1}] The process of creating the attack tree helped me better understand the attack scenario I selected
%   \item[\surveyq{LS-ADT4-L2}] I feel I could have achieved the same understanding by writing a text description of the attack.
%   \item[\surveyq{LS-ADT4-L3}] The ADT I created would help me communicate my threat scenario.
% \end{itemize}

% \subsubsection*{Short Response Questions}
% \begin{itemize}
%   \setlength{\itemindent}{\qIndent}
%   \item[\surveyq{LS-ADT4-W1}] What did you find easy about using ADTs?
%   \item[\surveyq{LS-ADT4-W2}] What did you find difficult about using ADT?\@
%   \item[\surveyq{LS-ADT4-W3}] Do you think ADTs have a place in the cybersecurity industry? If so, where? If not, why not?
%   \item[\surveyq{LS-ADT4-W4}] What aspects, if any, do you think are missing from ADTs?
%   \item[\surveyq{LS-ADT4-W5}] Do you hope to encounter ADTs in the future?
% \end{itemize}






