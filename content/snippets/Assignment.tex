\subsection{ADT 1: Assembling ADTs}

The following attack \textbf{leaf} nodes are provided. The overall goal of this scenario (and thus the root node of the tree) is \textbf{Rob bank}. Assemble an attack-defense tree using these leaf nodes. Do not add any additional leaf nodes. You may add any intermediary nodes you wish.

\textbf{Attack leaf nodes:}
% Why the fuck is there so much space here?
% \vspace{-10cm}
% \begin{itemize}
%   \item Hire Outright
%   \item Promise part of the stolen money
%   \item Threaten insiders

%   \item Buy tools
%   \item Steal tools
%   \item Gain Access
%   \item Walk through front door
%   \item Locate start of tunnel
%   \item Find direction to tunnel
% \end{itemize}

Hire Outright, Promise part of the stolen money, Threaten insiders, Buy tools, Steal tools, Gain Access, Walk through front door, Locate start of tunnel, Find direction to tunnel

\textbf{Defense leaf nodes:}

Personnel Risk Management, Check employee financial situation


\subsection*{Perception Questions}


\subsubsection{Likert Questions}
\begin{itemize}
  \setlength{\itemindent}{\qIndent}
  \item[\surveyq{LS-ADT1-L1}] I find the structure of attack tree easy to understand
  \item[\surveyq{LS-ADT1-L2}] Given all the nodes of an attack tree, it is easy for me to assemble the tree
  \item[\surveyq{LS-ADT1-L3}] Given only the leaf nodes of an attack tree, it is easy for me to assemble the tree.
  \item[\surveyq{LS-ADT1-L4}] I would rather define my own intermediary nodes
  \item[\surveyq{LS-ADT1-L5}] The process of assembling the attack tree helped me better understand the attack scenario.
\end{itemize}

\subsubsection{Short Response Questions}
\begin{itemize}
  \setlength{\itemindent}{\qIndent}
  \item[\surveyq{LS-ADT1-W1}] What did you find most difficult about this task? Why?
  \item[\surveyq{LS-ADT1-W2}] How did you go about solving this task? What was your methodology?
\end{itemize}



\subsection{ADT 2: Building ADTs}

The following text scenario is provided for you. Please create a complete attack defense tree \textbf{of this scenario}. \textbf{Do not add extra information that is not in the scenario}. Try to encapsulate the entire scenario with an attack-defense tree (don't leave any aspect of the attack scenario out).

\emph{Scenario:} 
The goal is to open a safe. To open the safe, an attacker can pick the lock,
learn the combination, cut open the safe, or install the safe improperly so
that he can easily open it later. Some models of safes are such that they cannot be picked, so if this model is used, then an attacker is unable to pick the lock. There are also auditing services to check if safes and other security technology is installed correctly. To learn the combination, the attacker
either has to find the combination written down or get the combination
from the safe owner. If the password is such that the safe owner can remember it, then the safe owner would not need to write it down.



\subsection*{Perception Questions}

\subsubsection{Likert Questions}
\begin{itemize}
  \setlength{\itemindent}{\qIndent}
  \item[\surveyq{LS-ADT2-L1}] I prefer reading attack trees to text descriptions of attacks.
  \item[\surveyq{LS-ADT2-L2}] The process of building the attack tree helped me better understand the attack scenario.
\end{itemize}

\subsubsection{Short Response Questions}
\begin{itemize}
  \setlength{\itemindent}{\qIndent}
  \item[\surveyq{LS-ADT2-W1}] What did you find most difficult about this task? Why?
  \item[\surveyq{LS-ADT2-W2}] How did you go about building the ADT?\@ What was your methodology?
  \item[\surveyq{LS-ADT2-W3}] What was the first node you added to your tree?
\end{itemize}


\subsection{ADT 3: Using Attack Trees}
This question is slightly different than the other questions. You need to create attack trees only. \textbf{This means you should NOT use defense nodes at all for this question}. In Part I, raw an attack tree from the provided scenario; \textbf{include all information from the scenario, do not include information that is not in the scenario}. In Part II, add onto the provided attack tree with new nodes and refinements that you find through your own research.

\subsection*{Part I: Attack tree from scenario}

\emph{Scenario:}  Many attackers aim to obtain personal data. Gathering personal data can be completed through unauthorized access to profile, credential creep, or a background data attack. Unauthorized access to profile requires gaining user credentials and accessing the profile. The credentials can be gained through a malware attack or a social engineering attack, and the profile can be accessed by stealing a phone or by remote access. Credential creep can be completed by submitting a request for additional data other than what is needed for verification or by user profiling. Finally, a background data attack requires both obtaining a sensitive dataset and linking the dataset via a request for verification. 


\subsection*{Part II: Finding new attack components}

Doing your own research, add at least 5 new nodes and 2 new refinements to the attack tree you created in the previous section.





\subsection*{Perception Questions}

\subsubsection{Likert Questions}
\begin{enumerate}
    \setlength{\itemindent}{\qIndent}
  \item[\surveyq{LS-ADT3-L1}] I prefer reading attack trees to text descriptions of attacks.
  \item[\surveyq{LS-ADT3-L2}] The process of building the attack tree helped me better understand the attack scenario.
  \item[\surveyq{LS-ADT3-L3}]  The ADT communicates the attack scenario better than the written scenario.
  \item[\surveyq{LS-ADT3-L4}] Using the ADT Web App made this task easier than if I had done it by hand.
\end{enumerate}

\subsubsection{Short Response Questions}
\begin{enumerate}
    \setlength{\itemindent}{\qIndent}
  \item[\surveyq{LS-ADT3-W1}] What did you find most difficult about this task? Why?
  \item[\surveyq{LS-ADT3-W2}] How did you go about building the ADT? What was your methodology?
  \item[\surveyq{LS-ADT3-W3}] What was the first node you added to your tree?
  \item[\surveyq{LS-ADT3-W4}]How would you describe using the ADT Web App? What aspects of the app made this task easier? What aspects made this task harder?
\end{enumerate}

\subsection{ADT 4: Creating ADTs}

Construct an attack defense tree of a scenario of your choice. Your tree should be complete (covers all reasonable attack scenarios) and reasonably large.


\subsection*{Perception Questions}

\subsubsection{Likert Questions}
\begin{itemize}
  \setlength{\itemindent}{\qIndent}
  \item[\surveyq{LS-ADT4-L1}] The process of creating the attack tree helped me better understand the attack scenario I selected
  \item[\surveyq{LS-ADT4-L2}] I feel I could have achieved the same understanding by writing a text description of the attack.
  \item[\surveyq{LS-ADT4-L3}] The ADT I created would help me communicate my threat scenario.
\end{itemize}

\subsubsection{Short Response Questions}
\begin{itemize}
  \setlength{\itemindent}{\qIndent}
  \item[\surveyq{LS-ADT4-W1}] What did you find easy about using ADTs?
  \item[\surveyq{LS-ADT4-W2}] What did you find difficult about using ADT?\@
  \item[\surveyq{LS-ADT4-W3}] Do you think ADTs have a place in the cybersecurity industry? If so, where? If not, why not?
  \item[\surveyq{LS-ADT4-W4}] What aspects, if any, do you think are missing from ADTs?
  \item[\surveyq{LS-ADT4-W5}] Do you hope to encounter ADTs in the future?
\end{itemize}